\documentclass[12pt]{article}

\usepackage{lipsum} % EXAMPLE only

\usepackage{amsmath} % Math
\usepackage[normalem]{ulem} % Strike-through Text
\usepackage{graphicx} % Images
\usepackage{titlesec} % Title Configuration

\graphicspath{ {./images/} }

% Command to strike-through text in math equations
\newcommand{\cross}[1]{\text{\sout{\ensuremath{#1}}}}
% Adjust the format of subsection (#. )
\renewcommand{\thesubsection}{\arabic{subsection}.\hspace{0.2em}}
% Change subsub sections numbering to alphabetical (a))
\renewcommand{\thesubsubsection}{\thesubsection\alph{subsubsection})}
% Configure sub and subsub sections to display inline
\setcounter{secnumdepth}{3}
\titleformat{\subsection}[runin]
  {\normalfont\normalsize\bfseries}{\thesubsection}{0em}{}
\titleformat{\subsubsection}[runin]
  {\normalfont\normalsize\bfseries}{\thesubsubsection}{0.5em}{}
% Create new commands to reference exercises
\newcommand{\exercise}{\subsection{}\setcounter{subsubsection}{0}}
\newcommand{\multipartexercise}{\addtocounter{subsection}{1}\setcounter{subsubsection}{0}}
\newcommand{\exercisepart}{\subsubsection{}}

\title{Assignment}
\author{Author}
\date{\today}

\begin{document}
\maketitle

\section*{Section/Chapter/Part Name if Required}
% Example of multipart exercise
% Every \multipartexercise starts a new exercise formatted as 1. 2. 3. ...
% Every \exercisepart adds a new sub-exercise for the last multipart exercise 1. a) 1. b) 1. c) ...
\multipartexercise \exercisepart
\lipsum[1-1] % EXAMPLE only

\begin{align*}
    a + b &= c \\
    c &= a + \pi \\
    \intertext{Explanation in between} \\
    b &= 42
\end{align*}

\exercisepart
\lipsum[2-2] % EXAMPLE only

\begin{figure}[h]
    \centering
    \includegraphics{filename}
    \caption{Image Caption}
    \label{imagereference}
\end{figure}

\multipartexercise \exercisepart
\lipsum[3-3] % EXAMPLE only

% A simple exercise without parts
\exercise
\begin{itemize}
    \begin{minipage}{\textwidth} % Prevents the content being broken up between pages
    \item Title of Item

    \lipsum[5-5] % EXAMPLE only
    \end{minipage}
\end{itemize}

\end{document}
