\documentclass[12pt]{article}

\usepackage{lipsum} % EXAMPLE only

% \usepackage[T1]{fontenc} % Encoding for more character support
% \usepackage[dvipsnames]{xcolor} % Colors
% \usepackage{enumitem} % More configurable enumerate
\usepackage{amsmath} % Math
% \usepackage[normalem]{ulem} % Strike-through Text
\usepackage{graphicx} % Images
\usepackage{titlesec} % Title configuration
% \usepackage{pifont} % More symbols

\graphicspath{ {./images/} }

% Command to strike-through text in math equations
\newcommand{\cross}[1]{\text{\sout{\ensuremath{#1}}}}
% Adjust the format of subsection (#. )
\renewcommand{\thesubsection}{\arabic{subsection}.\hspace{0.2em}}
% Change subsub sections numbering to alphabetical (a))
\renewcommand{\thesubsubsection}{\thesubsection\alph{subsubsection})}
% Configure sub and subsub sections to display inline
\setcounter{secnumdepth}{3}
\titleformat{\subsection}[runin]
  {\normalfont\normalsize\bfseries}{\thesubsection}{0em}{}
\titleformat{\subsubsection}[runin]
  {\normalfont\normalsize\bfseries}{\thesubsubsection}{0.5em}{}
% Create new commands to reference exercises
\newcommand{\exercise}{\subsection{}\setcounter{subsubsection}{0}}
\newcommand{\multipartexercise}{\addtocounter{subsection}{1}\setcounter{subsubsection}{0}}
\newcommand{\exercisepart}{\subsubsection{}}

% Command to strike-through text in math equations
% \newcommand{\cross}[1]{\text{\sout{\ensuremath{#1}}}}

\title{Assignment}
\author{Author}
\date{\today}

\begin{document}
\maketitle

\section*{Section/Chapter/Part Name if Required}
% Example of multipart exercise
% Every \multipartexercise starts a new exercise formatted as 1. 2. 3. ...
% Every \exercisepart adds a new sub-exercise for the last multipart exercise 1. a) 1. b) 1. c) ...
\multipartexercise \exercisepart
\lipsum[1-1] % EXAMPLE only

\begin{align*}
    a + b &= c \\
    c &= a + \pi \\
    \intertext{Explanation in between} \\
    b &= 42
\end{align*}

\begin{description}[labelwidth=0pt]
\item[]
\begin{DispWithArrows*}[format=cr,fleqn,mathindent=0pt]
    1 + 1 = 2 &
    \Arrow{Explanation for this step} \\
    2 + 2 \stackrel{?}{=} 5 &
\end{DispWithArrows*}
\end{description}

\exercisepart
\lipsum[2-2] % EXAMPLE only

\begin{figure}[h]
    \centering
    \includegraphics{filename}
    \caption{Image Caption}
    \label{imagereference}
\end{figure}

\multipartexercise \exercisepart
\lipsum[3-3] % EXAMPLE only

% A simple exercise without parts
\exercise
\begin{itemize}
    \begin{minipage}{\textwidth} % Prevents the content being broken up between pages
    \item Title of Item

    \lipsum[5-5] % EXAMPLE only
    \end{minipage}
\end{itemize}

% Example for a hexagon lattice with tikz - EXAMPLE only
\exercise
\begin{figure}[!h]
\centering
\begin{tikzpicture}[x=15mm,y=8.68mm]
  % Define colors for tikz 
  \definecolor{lightblue}{RGB}{173, 216, 230}
  \definecolor{lightgray}{RGB}{211, 211, 211}
  % Configure box node to be a hexagon
  \tikzset{
    box/.style={
      regular polygon,
      regular polygon sides=6,
      minimum size=20mm,
      inner sep=0mm,
      outer sep=0mm,
      rotate=0,
    draw
    }
  }

% Format at (2*horizontal+offset grid?,2*vertical+offset grid?)
% Fill with [box,fill=color]

\node[box] at (2*2+1,2*1+1) {1};
\node[box] at (2*2,2*1) {2};
\node[box] at (2*3,2*1) {3};
\node[box] at (2*2+1,2*0+1) {4};
\node[box] at (2*2,2*0) {5};
\node[box] at (2*3,2*0) {6};
\node[box] at (2*2+1,2*-1+1) {7};

\end{tikzpicture}
\caption{Hexagons}
\label{celllayout}
\end{figure}

\end{document}
